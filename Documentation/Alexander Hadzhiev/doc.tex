\documentclass[14pt]{extreport}

\usepackage[utf8]{inputenc}
\usepackage[bulgarian]{babel}

\usepackage{etoolbox,titling}

\usepackage{amsmath}
\usepackage{graphicx}
\usepackage[colorlinks=true, allcolors=blue]{hyperref}
\usepackage{geometry}
\geometry{verbose, tmargin=3cm, bmargin=3cm, lmargin=3cm, rmargin=3cm}

\setcounter{chapter}{-1}

\usepackage{caption}
\captionsetup[figure]{font=small}

\usepackage{hyperref}
\hypersetup{
    colorlinks=true,
    linkcolor=blue,
    filecolor=magenta,      
    urlcolor=cyan,
    pdftitle={Overleaf Example},
    pdfpagemode=FullScreen,
}

\usepackage{chngcntr}
\counterwithin{figure}{chapter}

\usepackage{scrextend}

\usepackage{minted}

\usepackage{listings}
\usepackage{xcolor}

\definecolor{keycolor}{rgb}{0,0.6,0}
\definecolor{backcolor}{rgb}{1, 1, 1}

\lstdefinestyle{mystyle}{
    backgroundcolor=\color{backcolor},   
	commentstyle=\color{gray},
    keywordstyle=\color{keycolor},
    numberstyle=\tiny\color{black},
    stringstyle=\color{red},
    basicstyle=\small\ttfamily\footnotesize,
    breakatwhitespace=false,         
    breaklines=true,                 
    captionpos=b,                    
    keepspaces=true,                 
    numbers=left,                    
    numbersep=5pt,                  
    showspaces=false,                
    showstringspaces=false,
    showtabs=false,                  
	tabsize=4
}

\lstset{style=mystyle}

\renewcommand{\lstlistingname}{Код}

\begin{document}

\chapter{Увод}

\chapter{Разработка на игри с Unreal Engine 4}

\chapter{Подбор на технологии за разработка на игри с Unreal Engine 4}

\section{Функционални изисквания към разработката}

Разработката на играта има за цел да постави основите на качествено разработен проект, който ще позволява лесно надграждане върху него.

\subsection{Изисквания към оръжията}
\begin{itemize}
	\item Параметри на оръжията, които позволяват висока разновидност.
	\item Функция за пълнител и презареждане на пълнителя.
	\item Функция за пиближаване (zoom).
	\item Различни режими на стрелба.
	\begin{itemize}
		\item Полуавтоматичен (Semi-automatic).
		\item Залпов (Burst).
		\item Автоматичен (Automatic).
	\end{itemize}
	\item Стрелба и откриване на удари чрез лъчево проследяване.
\end{itemize}

\subsection{Изисквания към системата за кръв}
\begin{itemize}
	\item Щит (Shield), които се възстановява за определено време, след поемане на удар.
	\item Живот (Кръв, Health), който не се възстановява от самосебе си.
	\item Предмет, който може да бъде използван от играча, ако животът му е под максималния.
\end{itemize}

\subsection{Изисквания към слотовете за оръжия}
\begin{itemize}
	\item Възможност за пазене на 2 оръжия и смяна между двете по време на игра.
\end{itemize}

\subsection{Изисквания към изкуствения интелект}
\begin{itemize}
	\item TO DO
\end{itemize}

\section{Избор на Unreal Engine 4}

\section{Избор на програмен език}

За разработка на игри в Unreal Engine 4 се използва C++. При стандартна работа със C++ в Unreal Engine се използват макрота, които позволяват по-лесна работа с класовете и дефиниране на типовете нужни за работа с обектите в Unreal Engine. 

\section{Избор на среда за разработка}

\subsection{Игрови двигател (Game Engine)}

За игрови двигател е, естествено, избран Unreal Engine, като по-точно версия 4.27. Разликата между версия 4.27 и 4.26 е най-вече във визуалните подобрения, но най-важното за тази версия, отнасящо се до разработката на проекта е новите алгоритми за компресиране на файлвоте нужини на Unreal Engine и файловете използвани от разработчиците на проекта. Това прави крайния продукт по-малкът от страна на размер, скорост на зареждане и предаване на мрежови пакети. 

Unreal Engine е избран вместо други игрови двигатели поради следните причини: 

\begin{itemize}
	\item Използва се C++ за разработка на функционалностите.
	\item Кодът му е свободно достъпен за всеки разработчик.
	\item Разработен е с 3D игри в предвид и позволява лесна работа в 3D пространството.
	\item Има всичко нужно за разработването на игри и в повечето случаи не се нуждае от външни добавки за да може да върши добра работа.
\end{itemize}

Разликата между Unreal Engine и други игрови двигатели, Unity например, е че Unreal Engine е направен за 3D игри. Unity е фокусиран върху 2D и мобилни игри, а Unreal Engine - към 3D игри за настолни компютри и конзоли. Разликата между Unreal Engine и други по-известни и големи игрови двигатели, е това че кодът на Unreal Engine е напълно отворен за четене от всеки разработчик на продукти с Unreal Engine.
Това позолява по-лесно откриване на проблеми в кода, по-добро разбиране базовите класове, върху които работят всички други части на Unreal Engine.

\subsection{IDE (Integrated Development Environment)}

За среда на разработка на прокета е избран Visual Studio 2019 Community Edition. Той се интегрира лесно с Unreal Engine, дебъгването с него става лесно и компилирането на проекта става сравнително бързо. IntelliSense, който е част от Visual Studio помага на по-бърза разработка, завършването на кода работи добре и се интегрира добре с начина на писане на код за Unreal Engine. 

\subsection{Среда за контрол на версиите}

За контрол на версиите е избран github, както и интеграцията му с git LFS (Large File Storage). Git LFS позволява качването на големи файлове, най-често бинарни, които се използват често при работата с Unreal Engine. Github позволява лесен начин за интегриране на нови свойства и промени на проекта, както и проверка на минали версии на проекта.

\chapter{Разработка на проекта}

\section{AMultiplayerFPSFirearm}

\section{AMultiplayerFPSCharacter}

\section{UMultiplayerFPSHealthSystem}

\section{AHealthPickup}

Този клас наследява AActor като дефинира предмет, който може да бъде поставен на картата и може да бъде взет от играча. Той се състои от два компонента - колизионна кутия и модел. За основен компонент е избран моделът, като кутията е прикрепна към основният компонент - моделът и позицията на кутията се смята спрямо моделът.

\lstinputlisting[language={[11]C++}, label={lst:hpickupdef}, caption=Дефиниция на класа AHealthPickup]{code/hpickupdef.h}

\lstinputlisting[language={[11]C++}, label={lst:hpickupcomp}, caption=Инициализация на компонентите в AHealthPickup]{code/hpickupcomp.cpp}

В код \ref{lst:hpickupdef} са дефинирани пет функции и един конструктор. В конструктора се инизиализират компонентите (показано в код \ref{lst:hpickupcomp}), BeginPlay се изпълнява след започване на tick на нивото, Tick се изпълнява на всеки кадър (frame), OnBeginOverlap се изпълнява след като кутията за колизия се пресече с друг обект за колизия, и Destroy функциите се викат съответно на сървъра и клиента. 

В BeginPlay се извиква макро за извикване на AddDynamic() върху динамични делегати за многократно предаване. 

\chapter{Ръководство на потребителя}

\section{Инсталиране на проекта}

\section{Работа с проекта}

\section{Стартиране на проекта}

\chapter{Заключение}

\tableofcontents

\end{document}
